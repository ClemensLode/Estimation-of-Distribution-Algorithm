\documentclass[12pt]{report}

\usepackage[ansinew]{inputenc}
\usepackage[ngerman]{babel}
\usepackage{a4, epsfig, fancyhdr}

\headheight = 15pt
\pagestyle{fancy}
\fancyhf{}
\chead{\sl\nouppercase\leftmark}
\cfoot{\thepage}


\begin{document}

%
% Deckblatt
%
\thispagestyle{empty}
\noindent

\begin{figure}[htbp]
  \begin{minipage}{0.6\textwidth}
  \raggedright
  	Institut f�r Angewandte Informatik\\
		und Formale Beschreibungsverfahren\\
		Universit�t Karlsruhe (TH)\\
		\vspace{1em}
		Forschungsgruppe Effiziente Algorithmen\\
		Prof. Dr. Hartmut Schmeck
  \end{minipage}\hfill
  \begin{minipage}{0.3\textwidth}
  	\raggedleft
  	\epsfig{file=AIFB_Logo_senkrecht1.eps, height=7em}
   \end{minipage}
\end{figure}

\vspace{3cm}

\noindent
Seminar "`Schwarmintelligenz"', Wintersemester 2005/06

\vspace{3cm}

\noindent
{\Large\textbf{Titel der Arbeit}}

\vspace{3cm}

\noindent
\begin{tabular}{ll}
Autor:	  &Vorname Nachname\\
Betreuer:	&Vorname Nachname
\end{tabular}

%
% Inhalts-, Abbildungs-, Tabellenverzeichnis
%
\tableofcontents
%\listoffigures
%\listoftables


%
% Die Seminarausarbeitung
%
\chapter{Das erste Kapitel}
Ich bin von Beruf Dachdecker. Am Tag des Unfalles arbeitete ich allein auf dem Dach eines sechsst�ckigen Neubaus. Als ich mit meiner Arbeit fertig war, hatte ich etwa 250kg Ziegel �brig.

\section{Ein Abschnitt}

Da ich sie nicht alle die Treppe hinunter tragen wollte, entschied ich mich daf�r, sie in einer Tonne an der Au�enseite des Geb�udes hinunterzulassen, die an einem Seil befestigt war, das �ber eine Rolle lief.

\begin{figure}[htbp]
	\center
  \epsfig{file=AIFB_Logo_senkrecht1.eps, width=1.2cm}
  \caption{\label{fig_1} Eine Abbildung.}
\end{figure}

Ich band also das Seil unten auf der Erde fest, ging auf das Dach und belud die Tonne. Dann ging ich wieder nach unten und band das Seil los. Ich hielt es fest, um die 250kg Ziegel\footnote{Eine Fu�note. Oder so \ldots} langsam herunterzulassen. Wenn Sie in Frage~11 des Unfallbericht-Formulars nachlesen, werden Sie feststellen, dass mein damaliges K�rpergewicht etwa 75kg betrug.

\subsection{Ein Unterabschnitt}

Da ich sehr �berrascht war, als ich pl�tzlich den Boden unter den F�ssen verlor und aufw�rts gezogen wurde, verlor ich meine Geistesgegenwart und verga� das Seil loszulassen. Ich glaube ich muss hier nicht sagen, dass ich mit immer gr��erer Geschwindigkeit am Geb�ude hinauf gezogen wurde. Man vergleiche hierzu auch Abbildung~\ref{fig_1} auf Seite~\pageref{fig_1}.

\chapter{Das zweite Kapitel}

Ich bin von Beruf Dachdecker. Am Tag des Unfalles arbeitete ich allein auf dem Dach eines sechsst�ckigen Neubaus. Als ich mit meiner Arbeit fertig war, hatte ich etwa 250kg Ziegel �brig. Da ich sie nicht alle die Treppe hinunter tragen wollte, ent-schied ich mich daf�r, sie in einer Tonne an der Au�enseite des Geb�udes hinunterzu-lassen, die an einem Seil befestigt war, das �ber eine Rolle lief. Ich band also das Seil unten auf der Erde fest, ging auf das Dach und belud die Tonne. Dann ging ich wieder nach unten und band das Seil los. Ich hielt es fest, um die 250kg Ziegel lang-sam herunterzulassen. Wenn Sie in Frage 11 des Unfallbericht-Formulars nachlesen, werden Sie feststellen, dass mein damaliges K�rpergewicht etwa 75kg betrug.

Da ich sehr �berrascht war, als ich pl�tzlich den Boden unter den F�ssen verlor und aufw�rts gezogen wurde, verlor ich meine Geistesgegenwart und verga� das Seil loszulassen. Ich glaube ich muss hier nicht sagen, dass ich mit immer gr��erer Ge-schwindigkeit am Geb�ude hinauf gezogen wurde.

Im Krankenhaus hatte ich Zeit, ein gutes Buch \cite{GKP94}, einen Zeitschriftenartikel \cite{BW03} und eine Konferenzver�ffentlichung \cite{FN02} zu lesen. Zudem stie� ich auf eine spannende Internetseite \cite{ips}.


%
% Literaturverzeichnis
% alternativ: BibTeX verwenden!
%
\addcontentsline{toc}{chapter}{Literaturverzeichnis}

\begin{thebibliography}{[XX]}
	\bibitem[1]{GKP94}  Graham R.\ L.; Knuth, D.\ E.; Patashnik, O.: \emph{Concrete Mathematics}. 
											Addison-Wesley, 2. Auflage, 1994.
	\bibitem[2]{BW03}  Bollig, B.; Wegener, I.: \emph{Functions that have read-once branching programs 
										 of quadratic size are not necessarily testable}. Information Processing Letters, 
										 Heftnummer 87, Seiten 25--29, 2003.
	\bibitem[3]{FN02}  Fischer, E.; Newman, I.: \emph{Functions that have read-twice constant width 
										 branching programs are not necessarily testable}. In: \emph{Proceedings of the 17th Conference on 	
										 Computational Complexity}, Seiten 73--79, 2002.
	\bibitem[4]{ips}   Lorem-Ipsum-Generator, \texttt{http://www.loremipsum.de}, Stand: 1. Dez. 2005
\end{thebibliography}
	
\end{document}